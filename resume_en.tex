\documentclass{resume_cover_letter}

\usepackage{xeCJK}
\setCJKmainfont[BoldFont={SimHei}]{SimSun}
\setCJKmonofont{SimHei}
\setCJKsansfont{Kaiti SC}

% infos
\firstname{朱震宇}
\familyname{}
\title{Financial Engineering Student}
\address{}{\faHome~福建--厦门}
\email{xmuzzy8@gmail.com}
%\homepage{www.gaelfoppolo.com}
\mobile{+86 18120775211}
\social[github]{zhuzhenyu666}
\social[twitter]{TiancaiMartin}
\extrainfo{驾照、证券从业资格证}
\photo[95pt][0pt]{img/1}	


\begin{document}

\maketitle

\vspace*{-8mm}

\section{\faGraduationCap~教育}

\tlcventry{2013}{2016}{厦门大学金融系金融工程}{本科}{\textit{G.P.A. 3.81/4}}{}{已修读课程39门,在校期间获得省级及以上科创竞赛奖项5项,校级科创竞赛奖项4项,奖学金2项,荣誉3项}

\section{\faBriefcase~经历}

\tlcventry{2013}{2015}{学生干部:厦门大学青年宣传中心秘书部副部长}{}{}{}{任职于机构办公室,负责机构财务运作,统筹安排财务预赛、决算、赞助及其他资金管理。任职期间参与组织活动包括厦门大学青年文化节、厦门大学宣传骨干培训等}

\tldatecventry{2015}{社会实践:“第二课堂 ”类志愿服务项目的开展现状及优化方式探究}{}{}{}{通过实地调研,对福建省本科高等院校学生参与支教项目的动机和限制因素进行数据分析,根据相关项目的开展现状和实施效果,从最大化志愿者激励角度提出优化支教类项目开展的方法。
\newline{实践团队获得厦门大学优秀实践队、厦门大学公共经济与政策大赛三等奖等表彰。}}

\section{\faGavel~项目}

\tldatecventry{2014}{Rock Paper Scissors Lizard Spock \faHandRockO~\faHandPaperO~\faHandScissorsO~\faHandLizardO~\faHandSpockO}{University Institute of Technology}{Montpellier, France}{\url{www.gaelfoppolo.com/projets/pfcls/}}{The project is based on the non-cooperative game \textit{Rock Paper Scissors Lizard Spock} by associating the age and sex of the player in order to bring to light the hidden correlations in the data or general gaming trends. The application propose to extract patterns of the form: \textit{`` after playing figure A followed by figure B, men aged between 18 and 22 years tend to play figure C ''}.
\newline{}}

\tldatecventry{2016}{Supervised learning}{Polytech}{Marseille}{\url{https://github.com/gaelfoppolo/examples-learning/}}{This project aims to find similarities in a set of formatted data and propose one or more characteristic(s) description(s) that carry the most information. The application produce solutions of the form: \textit{``a red square of size 8 is on a blue or green triangle of size 5, itself next-to a red circle of size 2 to 4''}.}


\section{\faNewspaperO~获奖与表彰}

\tldatecventry{2015/3}{全国大学生数学竞赛(非数学类)决赛全国二等奖}{}{}{}{}
\tldatecventry{2015/11}{2015年度厦门大学宏信奖学金、全国大学生统计建模大赛市场调查分析本科生组优秀奖}{}{}{}{}
\tldatecventry{2015/12}{2015年全国大学生数学建模福建省赛区一等奖}{}{}{}{}
\tldatecventry{2016/4}{2015年度厦门大学本科生科创竞赛暨德贞社会课堂三等奖}{}{}{}{}
\cvitem{其他}{Dean's List,校级优秀三好学生,厦门大学一等学业奖学金}

\section{\faGears~技能}

\begin{minipage}{\linewidth}
  \begin{multicols}{3}
    \centering{\textbf{编程语言\\}}
    \vspace*{2mm}
    \centering{Python~\textsc{\faStar~\faStar~\faStar~\faStarO~\faStarO}\\}
    \centering{Matlab~\textsc{\faStar~\faStar~\faStar~\faStarHalfO~\faStarO}\\}
    \centering{~~~R~~~\textsc{\faStar~\faStar~\faStar~\faStar~\faStarO}\\}
    \columnbreak
    \centering{\textbf{工具\\}}
    \vspace*{2mm}
    \centering{GitHub~~~~~~\textsc{\faStar~\faStar~\faStar~\faStarO~\faStarO}\\}
    \centering{\LaTeX~~~~~~~\textsc{\faStar~\faStar~\faStar~\faStarHalfO~\faStarO}\\}
    \centering{Office/VBA~\textsc{\faStar~\faStar~\faStar~\faStar~\faStarO}\\}
    \columnbreak
    \centering{\textbf{软件\\}}
    \vspace*{2mm}
    \centering{SPSS~~~\textsc{\faStar~\faStar~\faStar~\faStarHalfO~\faStarO}\\}
    \centering{Eviews~~\textsc{\faStar~\faStar~\faStar~\faStarHalfO~\faStarO}\\}
  \end{multicols}
\end{minipage}

\section{\faBeer~兴趣}

\tikzset{
    cercle/.pic={
      \node [draw, thick, circle, minimum width=10pt] {\tikzpictext};
    },
  }
%\hspace*{1mm}
\begin{minipage}{\linewidth}
  \begin{tikzpicture}
%  	\hspace*{20mm}
  	\pic [pic text={\huge \faLineChart}] {cercle};
    \node[draw=none] at (0,-1.1) {数据分析};
    \pic [pic text={\Huge \faDribbble}] at (20mm,0) {cercle};
    \node[draw=none] at (2,-1.1) {篮球};
    \pic [pic text={\Huge \faMusic}] at (40mm,0) {cercle};
    \node[draw=none] at (4,-1.1) {音乐};
    \pic [pic text={\Huge \faBook}] at (60mm,0) {cercle};
    \node[draw=none] at (6,-1.1) {阅读};
   \pic [pic text={\huge \faStackOverflow}] at (80mm,0) {cercle};
    \node[draw=none] at (8,-1.1) {编程};
    \pic [pic text={\Huge \faGamepad}] at (100mm,0) {cercle};
    \node[draw=none] at (10,-1.1) {2K};
    \pic [pic text={\huge \faBicycle}] at (120mm,0) {cercle};
    \node[draw=none] at (12,-1.1) {骑行};
    \pic [pic text={\huge  \faTrain}] at (140mm,0) {cercle};
    \node[draw=none] at (14,-1.1) {旅行 };
    \pic [pic text={\Huge \faGithub}] at (160mm,0) {cercle};
    \node[draw=none] at (16,-1.1) {共享};
  \end{tikzpicture}
\end{minipage}



\end{document}
